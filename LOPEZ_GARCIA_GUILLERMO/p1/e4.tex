Como algoritmo voraz que es, busca la mejor solución en cada paso,
pero no comprueba si en los siguientes pasos del problema sigue siendo
la mejor solución siguiendo cada uno de los caminos. Es decir,
busca la mejor solución en el menor tiempo posible.

He aquí la parte del algoritmo que más lo define como algoritmo voraz:

\lstset{language=C++, texcl=true}
\begin{lstlisting}[frame=single]
    List<Defense*>::iterator currentDefense = defenses.begin(); // Se instancia todas las posiciones de las defensas a un
                                                                // valor negativo para indicar que no estan colocadas.
    while(++currentDefense != defenses.end())
    {
        while(!valoracionesCeldas.empty())
        {
            int row = valoracionesCeldas.rbegin()->i_;
            int col = valoracionesCeldas.rbegin()->j_;

            // Se comprueba que esa celda sea factible para colocar la defensa (como es la primera, la base)
            if(factibilidad(row, col, nCellsWidth, nCellsHeight, mapWidth, mapHeight,
                            obstacles, defenses, (*currentDefense)))
            {
                Vector3 v = cellCenterToPosition(row, col, cellWidth, cellHeight);
                (*currentDefense)->position.x = v.x;
                (*currentDefense)->position.y = v.y;
                (*currentDefense)->position.z = 0;
                break;
            }
            valoracionesCeldas.pop_back(); // Se actualiza el vector de valoraciones
            std::sort(valoracionesCeldas.begin(), valoracionesCeldas.end());
        }
    }
\end{lstlisting}

Como se puede apreciar, coloca cada defensa según la recibe de la lista, pero
no se preocupa de las propias caracteristas de la defensa o como al colocarla
antes que otra puede afectar a la calidad de la defensa de la base.

Por tanto, busca la mejor solución en el menor coste de tiempo posible. Es
un algoritmo voraz.
