La función para otorgar un determinado valor a cada una de las
celdas se basa en dar mayor puntuación a las celdas cuyo centro
esta libre y además, se encuentren mas cerca del centro del
tablero. En caso contrario a que su centro este libre, reciben
una puntuación de 0. Cuanto mas lejos este la celda a puntar
del centro, en cualquiera de los 4 sentidos, recibirá menor
puntuación.

Por otra parte, al evaluar todas las celdas así, no distingo
si el centro de extracción ya ha sido colocado o no. Tan solo
evalúo según las distancia con el centro del terreno.

A continuación, expongo la función descrita llamada cellValue:

\lstset{language=C++, texcl=true}
\begin{lstlisting}[frame=single]
float cellValue(int row, int col, bool** freeCells, int nCellsWidth, int nCellsHeight,
                float mapWidth, float mapHeight, List<Object*> obstacles, List<Defense*> defenses,
                bool isBase = false)
{
    try // Por si acaso freeCells no es del tamaño esperado, se recoje la excepcion y se devuelve una valoracion negativa
    {
        if (row >= 0 && col >= 0) // Comprueba si la fila y columna es mayor que cero
        {
            if(row < nCellsWidth && col < nCellsHeight) // Comprueba que la fila y columna este dentro de los margenes del trablero
            {
                if(freeCells[row][col]) // Comprobacion si el centro de la celda esta disponible
                {
                    int celdaCentroY = nCellsWidth / 2;
                    int celdaCentroX = nCellsHeight / 2;

                    float a = 0.0; // Valoracion de la casilla central en el eje Y
                    if(row >= celdaCentroY)
                        a = celdaCentroY*2 - row;
                    else if(row < celdaCentroY)
                        a = row;

                    float b = 0.0; // Valoracion de la casilla central en el eje X
                    if(col >= celdaCentroX)
                        b = celdaCentroX*2 - col;
                    else if(col < celdaCentroX)
                        b = col;

                    return a * b; // La mejor valoracion es el centro del tablero
                } else return 0.0;
            }
        }
    } catch(std::exception &e) {}
    return -1; // Por si se manda a valorar una celda que no esta dentro del tablero
}
\end{lstlisting}

Como es obvio, le paso un parametro para saber si voy a evaluar para colocar la base
o el resto de las defensas, pero eso se utilizará en futuras versiones del algoritmo.
