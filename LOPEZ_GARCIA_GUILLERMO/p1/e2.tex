Respecto a la función de factibilidad, básicamente se comprueba
si la defensa a colocar choca con algún obstaculo o con alguna
otra defensa colocada. Así mismo, también comprueba que la defensa
a colocar no se salga de los bordes del tablero.

La única contrariedad en el desarrollo de está función es que
no cumple con la programación estructurada, pero porque he decidido
no programarla así. Considerando motivos de eficiencia y aprovechando
la capacidad del C++ con su orientación a objetos y su capacidad
de programación no estructurada, la función tiene varios puntos de
retorno según las condiciones que se cumplan.

A continuación, el código de la función descrita:

\lstset{language=C++, texcl=true}
\begin{lstlisting}[frame=single]
bool factibilidad(int row, int col, int nCellsWidth, int nCellsHeight, float mapWidth, float mapHeight,
                  List<Object*> obstacles, List<Defense*> defenses, Object* defense = NULL)
{
    if(row >= 0 && row <= nCellsWidth && col >= 0 && col <= nCellsHeight)
    {
        float cellWidth = mapWidth / nCellsWidth;
        float cellHeight = mapHeight / nCellsHeight;

        Vector3 pf = cellCenterToPosition(row, col, cellWidth, cellHeight);

        if(defense != NULL)
        {
            List<Object*>::iterator o = obstacles.begin();
            while (o != obstacles.end())
            {
                float xObstacle = (*o)->position.x;
                float yObstacle = (*o)->position.y;
                Vector3 po = Vector3(xObstacle, yObstacle, 0);

                if(_distance(pf, po) < (defense->radio + (*o)->radio)) return false;
                ++o;
            }

            List<Defense*>::iterator d = defenses.begin();
            while (d != defenses.end())
            {
                if(defense == (*d)) { ++d; continue; } // Si estamos comparando una defensa con ella misma

                float xDefense = (*d)->position.x;
                float yDefense = (*d)->position.y;

                if(xDefense < 0 || yDefense < 0) { ++d; continue; } // Si la defensa no esta colocada, no se compara con ella

                Vector3 pd = Vector3(xDefense, yDefense, 0);

                if(_distance(pf, pd) < (defense->radio + (*d)->radio)) return false;
                ++d;
            }

            // Si la defensa se sale por los bordes
            if(pf.x + defense->radio > mapWidth || pf.x - defense->radio < 0) return false;
            if(pf.y + defense->radio > mapHeight || pf.y - defense->radio < 0) return false;

            return true; // Para todas las condiciones
        }
    }
    return false;
}
\end{lstlisting}
