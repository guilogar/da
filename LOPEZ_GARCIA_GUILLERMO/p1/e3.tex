El algoritmo voraz que resuelve el problema es el siguiente:

\lstset{language=C++, texcl=true}
\begin{lstlisting}[frame=single]
    float cellWidth = mapWidth / nCellsWidth;
    float cellHeight = mapHeight / nCellsHeight;

    // Vector de valoraciones para la base
    std::vector<Valoracion> valoracionesCeldas =
        obtenerValoraciones(freeCells, nCellsWidth, nCellsHeight,
                            mapWidth, mapHeight, obstacles, defenses, true);

    //if(!defenses.empty())
    List<Defense*>::iterator currentDefense = defenses.begin(); // Se instancia todas las posiciones de las defensas a un
                                                                // valor negativo para indicar que no estan colocadas.
    while(currentDefense != defenses.end())
    {
        (*currentDefense)->position.x = -1;
        (*currentDefense)->position.y = -1;
        (*currentDefense)->position.z = -1;
        ++currentDefense;
    }

    currentDefense = defenses.begin(); // Se actualiza el iterador al principio
    while(!valoracionesCeldas.empty()) // Mientras que hayas celdas disponibles
    {
        int row = valoracionesCeldas.rbegin()->i_;
        int col = valoracionesCeldas.rbegin()->j_;

        // Se comprueba que esa celda sea factible para colocar la defensa (como es la primera, la base)
        if(factibilidad(row, col, nCellsWidth, nCellsHeight, mapWidth, mapHeight, obstacles, defenses, (*currentDefense)))
        {
            Vector3 v = cellCenterToPosition(row, col, cellWidth, cellHeight);
            (*currentDefense)->position.x = v.x;
            (*currentDefense)->position.y = v.y;
            (*currentDefense)->position.z = 0;
            break;
        }
        valoracionesCeldas.pop_back(); // Se actualiza el vector de valoraciones
    }
\end{lstlisting}

Para poder utilizar abstraer el concepto de valorar una celda,
he decidido crear una estructura llamada Valoracion que encierra
el valor de una valoracion y las posiciones de la misma dentro del
tablero.

He aquí la definición de la misma:

\lstset{language=C++, texcl=true}
\begin{lstlisting}[frame=single]
struct Valoracion // Estructura para guardar una valoracion, con la fila y columna que se hace
{
    float value_;
    int i_, j_;
    Valoracion(int i = 0, int j = 0, int value = 0.0) : i_(i), j_(j), value_(value) {}
    bool operator < (const Valoracion& v) { return ( value_ < v.value_ ); }
};
\end{lstlisting}

Además, redefino el operador < de la estructura para posteriormente, poder ordernar
un vector de este tipo.

A continuación, se muestra la función que devuelve un vector de valoraciones
ordenados de menor a mayor, así pues, el orden de escojer la mayor valoración
es siempre de O(1):

\lstset{language=C++, texcl=true}
\begin{lstlisting}[frame=single]
std::vector<Valoracion> obtenerValoraciones(bool** freeCells, int nCellsWidth, int nCellsHeight,
                                            float mapWidth, float mapHeight, std::list<Object*> obstacles,
                                            std::list<Defense*> defenses, bool isBase = false)
{
    std::vector<Valoracion> valoracionesCeldas; // Vector de valoraciones

    for (int i = 0; i < nCellsHeight; i++)
    {
        for (int j = 0; j < nCellsWidth; j++)
        {
            float v = cellValue(i, j, freeCells, nCellsWidth, nCellsHeight,
                                mapWidth, mapHeight, obstacles, defenses,
                                isBase);
            valoracionesCeldas.push_back(Valoracion(i, j, v)); // Se valora una celda
        }
    }

    std::sort(valoracionesCeldas.begin(), valoracionesCeldas.end()); // Se ordena el vector de mayor a menor valoracion

    return valoracionesCeldas;
}
\end{lstlisting}
