He aquí el algoritmo completo, tanto para colocar la primera defensa,
centro de extracción, como para colocar el resto de las defensas:

\lstset{language=C++, texcl=true}
\begin{lstlisting}[frame=single]
void DEF_LIB_EXPORTED placeDefenses(bool** freeCells, int nCellsWidth, int nCellsHeight, float mapWidth,
                                    float mapHeight, std::list<Object*> obstacles, std::list<Defense*> defenses)
{
    float cellWidth = mapWidth / nCellsWidth;
    float cellHeight = mapHeight / nCellsHeight;

    // Vector de valoraciones para la base
    std::vector<Valoracion> valoracionesCeldas =
        obtenerValoraciones(freeCells, nCellsWidth, nCellsHeight,
                            mapWidth, mapHeight, obstacles, defenses, true);

    //if(!defenses.empty())
    List<Defense*>::iterator currentDefense = defenses.begin(); // Se instancia todas las posiciones de las defensas a un
                                                                // valor negativo para indicar que no estan colocadas.
    while(currentDefense != defenses.end())
    {
        (*currentDefense)->position.x = -1;
        (*currentDefense)->position.y = -1;
        (*currentDefense)->position.z = -1;
        ++currentDefense;
    }

    currentDefense = defenses.begin(); // Se actualiza el iterador al principio
    while(!valoracionesCeldas.empty()) // Mientras que hayas celdas disponibles
    {
        int row = valoracionesCeldas.rbegin()->i_;
        int col = valoracionesCeldas.rbegin()->j_;

        // Se comprueba que esa celda sea factible para colocar la defensa (como es la primera, la base)
        if(factibilidad(row, col, nCellsWidth, nCellsHeight, mapWidth, mapHeight, obstacles, defenses, (*currentDefense)))
        {
            Vector3 v = cellCenterToPosition(row, col, cellWidth, cellHeight);
            (*currentDefense)->position.x = v.x;
            (*currentDefense)->position.y = v.y;
            (*currentDefense)->position.z = 0;
            break;
        }
        valoracionesCeldas.pop_back(); // Se actualiza el vector de valoraciones
    }

    // Vector de valoraciones para el resto de las defensas
    valoracionesCeldas = obtenerValoraciones(freeCells, nCellsWidth, nCellsHeight,
                                              mapWidth, mapHeight, obstacles, defenses);

    while(++currentDefense != defenses.end())
    {
        while(!valoracionesCeldas.empty())
        {
            int row = valoracionesCeldas.rbegin()->i_;
            int col = valoracionesCeldas.rbegin()->j_;

            // Se comprueba que esa celda sea factible para colocar la defensa (como es la primera, la base)
            if(factibilidad(row, col, nCellsWidth, nCellsHeight, mapWidth, mapHeight,
                            obstacles, defenses, (*currentDefense)))
            {
                Vector3 v = cellCenterToPosition(row, col, cellWidth, cellHeight);
                (*currentDefense)->position.x = v.x;
                (*currentDefense)->position.y = v.y;
                (*currentDefense)->position.z = 0;
                break;
            }
            valoracionesCeldas.pop_back(); // Se actualiza el vector de valoraciones
            std::sort(valoracionesCeldas.begin(), valoracionesCeldas.end());
        }
    }

    #ifdef PRINT_DEFENSE_STRATEGY

        float** cellValues = new float* [nCellsHeight];
        for(int i = 0; i < nCellsHeight; ++i) {
           cellValues[i] = new float[nCellsWidth];
           for(int j = 0; j < nCellsWidth; ++j) {
               cellValues[i][j] = ((int)(cellValue(i, j))) % 256;
           }
        }

        dPrintMap("strategy.ppm", nCellsHeight, nCellsWidth, cellHeight, cellWidth,
                  freeCells , cellValues, std::list<Defense*>(), true);

        for(int i = 0; i < nCellsHeight; ++i) delete [] cellValues[i];
        delete [] cellValues;
        cellValues = NULL;

    #endif
}
\end{lstlisting}
