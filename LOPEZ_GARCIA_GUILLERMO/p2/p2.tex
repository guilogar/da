\documentclass[]{article}

\usepackage[left=2.00cm, right=2.00cm, top=2.00cm, bottom=2.00cm]{geometry}
\usepackage[spanish,es-noshorthands]{babel}
\usepackage[utf8]{inputenc} % para tildes y ñ
\usepackage{graphicx} % para las figuras
\usepackage{xcolor}
\usepackage{listings} % para el código fuente en c++

\lstdefinestyle{customc}{
  belowcaptionskip=1\baselineskip,
  breaklines=true,
  frame=single,
  xleftmargin=\parindent,
  language=C++,
  showstringspaces=false,
  basicstyle=\footnotesize\ttfamily,
  keywordstyle=\bfseries\color{green!40!black},
  commentstyle=\itshape\color{gray!40!gray},
  identifierstyle=\color{black},
  stringstyle=\color{orange},
}
\lstset{style=customc}


%opening
\title{Práctica 2. Programación dinámica}
\author{Guillermo López García \\ % mantenga las dos barras al final de la línea y este comentario
guillermo.lopezgarcia@alum.uca.es \\ % mantenga las dos barras al final de la línea y este comentario
Teléfono: +34 651 737 456 \\ % mantenga las dos barras al final de la linea y este comentario
NIF: 49340296Y \\ % mantenga las dos barras al final de la línea y este comentario
}


\begin{document}

\maketitle

%\begin{abstract}
%\end{abstract}

% Ejemplo de ecuación a trozos
%
%$f(i,j)=\left\{ 
%  \begin{array}{lcr}
%      i + j & si & i < j \\ % caso 1
%      i + 7 & si & i = 1 \\ % caso 2
%      2 & si & i \geq j     % caso 3
%  \end{array}
%\right.$

\begin{enumerate}
\item Formalice a continuación y describa la función que asigna un determinado valor a cada uno de los tipos de defensas.

La estructura de datos elegida para representar el mapa es un vector de Valoraciones,
donde cada Valoracion es a su vez una estructura clásica del c++ que posee comportamiento
de clase y sobrecarga de operadores.

Aquí el código que especifica dicha estructura:

\lstset{language=C++, texcl=true}
\begin{lstlisting}[frame=single]
struct Valoracion // Estructura para guardar una valoracion, con la fila y columna que se hace
{
    float value_;
    int i_, j_;
    Valoracion(int i = 0, int j = 0, int value = 0.0) : i_(i), j_(j), value_(value) {}
    bool operator < (const Valoracion& v) { return ( value_ < v.value_ ); }
};
\end{lstlisting}


\item Describa la estructura o estructuras necesarias para representar la tabla de subproblemas resueltos.

\lstset{language=C++, texcl=true}
\begin{lstlisting}[frame=single]
std::vector<Valoracion> ordenacionPorFusion(std::vector<Valoracion>& vector)
{
    if(vector.size() <= 1) return vector;
    else if(vector.size() == 2)
    {
        if(vector[0] < vector[1])
        {
            Valoracion v = vector[0];
            vector[0] = vector[1];
            vector[1] = v;
        }
    }
    else
    {
        int tamanio = vector.size();
        int mitad = tamanio / 2;
        std::vector<Valoracion> primeraMitad;
        std::vector<Valoracion> segundaMitad;
        
        for (int i = 0; i < mitad; i++) { primeraMitad.push_back(vector[i]); }
        for (int i = mitad; i < tamanio; i++) { segundaMitad.push_back(vector[i]); }
        
        primeraMitad = ordenacionPorFusion(primeraMitad);
        segundaMitad = ordenacionPorFusion(segundaMitad);
        vector.clear();
        
        for (int i = 0; i < primeraMitad.size(); i++)
            vector.push_back(primeraMitad[i]);
        
        for (int i = 0; i < segundaMitad.size(); i++)
            vector.push_back(segundaMitad[i]);
    }
    
    return vector;
}
\end{lstlisting}


\item En base a los dos ejercicios anteriores, diseñe un algoritmo que determine el máximo beneficio posible a obtener dada una combinación de defensas y \emph{ases} disponibles. Muestre a continuación el código relevante.

\begin{lstlisting}
// sustituya este codigo por su respuesta
void selectDefenses(...) {

    unsigned int cost = 0;
    std::list<Defense*>::iterator it = defenses.begin();
    while(it != defenses.end()) {
        if(cost + (*it)->cost <= ases) {
            selectedIDs.push_back((*it)->id);
            cost += (*it)->cost;
        }
        ++it;
    }
}
\end{lstlisting}

\item Diseñe un algoritmo que recupere la combinación óptima de defensas a partir del contenido de la tabla de subproblemas resueltos. Muestre a continuación el código relevante.

\begin{lstlisting}
// sustituya este codigo por su respuesta
void selectDefenses(...) {

    unsigned int cost = 0;
    std::list<Defense*>::iterator it = defenses.begin();
    while(it != defenses.end()) {
        if(cost + (*it)->cost <= ases) {
            selectedIDs.push_back((*it)->id);
            cost += (*it)->cost;
        }
        ++it;
    }
}
\end{lstlisting}

\end{enumerate}

Todo el material incluido en esta memoria y en los ficheros asociados es de mi autoría o ha sido facilitado por los profesores de la asignatura. Haciendo entrega de este documento confirmo que he leído la normativa de la asignatura, incluido el punto que respecta al uso de material no original.

\end{document}
