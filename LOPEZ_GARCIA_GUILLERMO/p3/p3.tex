\documentclass[]{article}

\usepackage[left=2.00cm, right=2.00cm, top=2.00cm, bottom=2.00cm]{geometry}
\usepackage[spanish,es-noshorthands]{babel}
\usepackage[utf8]{inputenc} % para tildes y ñ
\usepackage{graphicx} % para las figuras
\usepackage{xcolor}
\usepackage{listings} % para el código fuente en c++

\lstdefinestyle{customc}{
  belowcaptionskip=1\baselineskip,
  breaklines=true,
  frame=single,
  xleftmargin=\parindent,
  language=C++,
  showstringspaces=false,
  basicstyle=\footnotesize\ttfamily,
  keywordstyle=\bfseries\color{green!40!black},
  commentstyle=\itshape\color{gray!40!gray},
  identifierstyle=\color{black},
  stringstyle=\color{orange},
}
\lstset{style=customc}


%opening
\title{Práctica 3. Divide y vencerás}
\author{Guillermo López García \\ % mantenga las dos barras al final de la línea y este comentario
guillermo.lopezgarcia@alum.uca.es \\ % mantenga las dos barras al final de la línea y este comentario
Teléfono: +34 651 737 456 \\ % mantenga las dos barras al final de la linea y este comentario
NIF: 49340296Y \\ % mantenga las dos barras al final de la línea y este comentario
}


\begin{document}

\maketitle

%\begin{abstract}
%\end{abstract}

% Ejemplo de ecuación a trozos
%
%$f(i,j)=\left\{ 
%  \begin{array}{lcr}
%      i + j & si & i < j \\ % caso 1
%      i + 7 & si & i = 1 \\ % caso 2
%      2 & si & i \geq j     % caso 3
%  \end{array}
%\right.$

\begin{enumerate}
\item Describa las estructuras de datos utilizados en cada caso para la representación del terreno de batalla. 

La estructura de datos elegida para representar el mapa es un vector de Valoraciones,
donde cada Valoracion es a su vez una estructura clásica del c++ que posee comportamiento
de clase y sobrecarga de operadores.

Aquí el código que especifica dicha estructura:

\lstset{language=C++, texcl=true}
\begin{lstlisting}[frame=single]
struct Valoracion // Estructura para guardar una valoracion, con la fila y columna que se hace
{
    float value_;
    int i_, j_;
    Valoracion(int i = 0, int j = 0, int value = 0.0) : i_(i), j_(j), value_(value) {}
    bool operator < (const Valoracion& v) { return ( value_ < v.value_ ); }
};
\end{lstlisting}


\item Implemente su propia versión del algoritmo de ordenación por fusión. Muestre a continuación el código fuente relevante. 

\lstset{language=C++, texcl=true}
\begin{lstlisting}[frame=single]
std::vector<Valoracion> ordenacionPorFusion(std::vector<Valoracion>& vector)
{
    if(vector.size() <= 1) return vector;
    else if(vector.size() == 2)
    {
        if(vector[0] < vector[1])
        {
            Valoracion v = vector[0];
            vector[0] = vector[1];
            vector[1] = v;
        }
    }
    else
    {
        int tamanio = vector.size();
        int mitad = tamanio / 2;
        std::vector<Valoracion> primeraMitad;
        std::vector<Valoracion> segundaMitad;
        
        for (int i = 0; i < mitad; i++) { primeraMitad.push_back(vector[i]); }
        for (int i = mitad; i < tamanio; i++) { segundaMitad.push_back(vector[i]); }
        
        primeraMitad = ordenacionPorFusion(primeraMitad);
        segundaMitad = ordenacionPorFusion(segundaMitad);
        vector.clear();
        
        for (int i = 0; i < primeraMitad.size(); i++)
            vector.push_back(primeraMitad[i]);
        
        for (int i = 0; i < segundaMitad.size(); i++)
            vector.push_back(segundaMitad[i]);
    }
    
    return vector;
}
\end{lstlisting}



\item Implemente su propia versión del algoritmo de ordenación rápida. Muestre a continuación el código fuente relevante. 

\begin{lstlisting}
// sustituya este codigo por su respuesta
void selectDefenses(...) {

    unsigned int cost = 0;
    std::list<Defense*>::iterator it = defenses.begin();
    while(it != defenses.end()) {
        if(cost + (*it)->cost <= ases) {
            selectedIDs.push_back((*it)->id);
            cost += (*it)->cost;
        }
        ++it;
    }
}
\end{lstlisting}

\item Realice pruebas de caja negra para asegurar el correcto funcionamiento de los algoritmos de ordenación implementados en los ejercicios anteriores. Detalle a continuación el código relevante.

\begin{lstlisting}
// sustituya este codigo por su respuesta
void selectDefenses(...) {

    unsigned int cost = 0;
    std::list<Defense*>::iterator it = defenses.begin();
    while(it != defenses.end()) {
        if(cost + (*it)->cost <= ases) {
            selectedIDs.push_back((*it)->id);
            cost += (*it)->cost;
        }
        ++it;
    }
}
\end{lstlisting}

\item Analice de forma teórica la complejidad de las diferentes versiones del algoritmo de colocación de defensas en función de la estructura de representación del terreno de batalla elegida. Comente a continuación los resultados. Suponga un terreno de batalla cuadrado en todos los casos. 

Escriba aquí su respuesta al ejercicio 5.

\item Incluya a continuación una gráfica con los resultados obtenidos. Utilice un esquema indirecto de medida (considere un error absoluto de valor 0.01 y un error relativo de valor 0.001). Considere en su análisis los planetas con códigos 1500, 2500, 3500,..., 10500. Incluya en el análisis los planetas que considere oportunos para mostrar información relevante.

Escriba aquí su respuesta al ejercicio 6.

\end{enumerate}

Todo el material incluido en esta memoria y en los ficheros asociados es de mi autoría o ha sido facilitado por los profesores de la asignatura. Haciendo entrega de este documento confirmo que he leído la normativa de la asignatura, incluido el punto que respecta al uso de material no original.

\end{document}
