\lstset{language=C++, texcl=true}
\begin{lstlisting}[frame=single]
std::vector<Valoracion> obtenerValoraciones(bool** freeCells, int nCellsWidth, int nCellsHeight,
                                            float mapWidth, float mapHeight, std::list<Object*> obstacles,
                                            std::list<Defense*> defenses, bool isBase = false,
                                            unsigned int modOrdenacion = 0)
{
    std::vector<Valoracion> valoracionesCeldas; // Vector de valoraciones
    
    for (int i = 0; i < nCellsHeight; i++)
    {
        for (int j = 0; j < nCellsWidth; j++)
        {
            float v = defaultCellValue(i, j, freeCells, nCellsWidth, nCellsHeight,
                                       mapWidth, mapHeight, obstacles, defenses,
                                       isBase);
            valoracionesCeldas.push_back(Valoracion(i, j, v)); // Se valora una celda
        }
    }
    
    switch (modOrdenacion) {
        case 0: break; // Sin ordenación
        case 1:
            {
                // Ordenacion por fusion
                valoracionesCeldas = ordenacionPorFusion(valoracionesCeldas);
            } break;
        case 2:
            {
                // Ordenacion rapida
                valoracionesCeldas = ordenacionRapida(valoracionesCeldas);
            } break;
        case 3:
            {
                // Ordenacion por monticulo
                std::make_heap(valoracionesCeldas.begin(), valoracionesCeldas.end());
            } break;
        default:
            std::sort(
                    valoracionesCeldas.begin(),
                    valoracionesCeldas.end()
            ); // Se ordena el vector de menor a mayor valoracion
    }
    
    return valoracionesCeldas;
}
\end{lstlisting}
