\lstset{language=C++, texcl=true}
\begin{lstlisting}[frame=single]
int obtenerPivote(std::vector<Valoracion> vector)
{
    int p = 0;
    
    if(vector.size() > 0)
    {
        Valoracion v = vector[0];
        for (int k = 1; k < vector.size(); k++)
        {
            if(v < vector[k])
            {
                Valoracion va = vector[p];
                vector[p] = vector[k];
                vector[k] = va;
                p++;
            }
        }
        vector[0] = vector[p];
        vector[p] = v;
    }
    return p;
}

std::vector<Valoracion> ordenacionRapida(std::vector<Valoracion>& vector)
{
    if(vector.size() <= 1) return vector;
    else if(vector.size() == 2)
    {
        if(vector[0] < vector[1])
        {
            Valoracion v = vector[0];
            vector[0] = vector[1];
            vector[1] = v;
        }
    }
    else
    {
        int tamanio = vector.size();
        int pivote = obtenerPivote(vector);
        if(pivote == 0) pivote = tamanio / 2;
        
        std::vector<Valoracion> primeraMitad;
        std::vector<Valoracion> segundaMitad;
        
        for (int i = 0; i < pivote; i++) { primeraMitad.push_back(vector[i]); }
        for (int i = pivote; i < tamanio; i++) { segundaMitad.push_back(vector[i]); }
        
        primeraMitad = ordenacionRapida(primeraMitad);
        segundaMitad = ordenacionRapida(segundaMitad);
        vector.clear();
        
        for (int i = 0; i < primeraMitad.size(); i++)
            vector.push_back(primeraMitad[i]);
        
        for (int i = 0; i < segundaMitad.size(); i++)
            vector.push_back(segundaMitad[i]);
    }
    
    return vector;
}
\end{lstlisting}
