\documentclass[]{article}

\usepackage[left=2.00cm, right=2.00cm, top=2.00cm, bottom=2.00cm]{geometry}
\usepackage[spanish,es-noshorthands]{babel}
\usepackage[utf8]{inputenc} % para tildes y ñ
\usepackage{graphicx} % para las figuras
\usepackage{xcolor}
\usepackage{listings} % para el código fuente en c++

%opening
\title{Práctica 4. Exploración de grafos}
\author{Guillermo López García \\ % mantenga las dos barras al final de la línea y este comentario
guillermo.lopezgarcia@alum.uca.es \\ % mantenga las dos barras al final de la línea y este comentario
Teléfono: +34 651 737 456 \\ % mantenga las dos barras al final de la linea y este comentario
NIF: 49340296Y \\ % mantenga las dos barras al final de la línea y este comentario
}


\begin{document}

\maketitle

%\begin{abstract}
%\end{abstract}

% Ejemplo de ecuación a trozos
%
%$f(i,j)=\left\{ 
%  \begin{array}{lcr}
%      i + j & si & i < j \\ % caso 1
%      i + 7 & si & i = 1 \\ % caso 2
%      2 & si & i \geq j     % caso 3
%  \end{array}
%\right.$

\begin{enumerate}
\item Comente el funcionamiento del algoritmo y describa las estructuras necesarias para llevar a cabo su implementación.

La estructura de datos elegida para representar el mapa es un vector de Valoraciones,
donde cada Valoracion es a su vez una estructura clásica del c++ que posee comportamiento
de clase y sobrecarga de operadores.

Aquí el código que especifica dicha estructura:

\lstset{language=C++, texcl=true}
\begin{lstlisting}[frame=single]
struct Valoracion // Estructura para guardar una valoracion, con la fila y columna que se hace
{
    float value_;
    int i_, j_;
    Valoracion(int i = 0, int j = 0, int value = 0.0) : i_(i), j_(j), value_(value) {}
    bool operator < (const Valoracion& v) { return ( value_ < v.value_ ); }
};
\end{lstlisting}


\item Incluya a continuación el código fuente relevante del algoritmo.

\lstset{language=C++, texcl=true}
\begin{lstlisting}[frame=single]
std::vector<Valoracion> ordenacionPorFusion(std::vector<Valoracion>& vector)
{
    if(vector.size() <= 1) return vector;
    else if(vector.size() == 2)
    {
        if(vector[0] < vector[1])
        {
            Valoracion v = vector[0];
            vector[0] = vector[1];
            vector[1] = v;
        }
    }
    else
    {
        int tamanio = vector.size();
        int mitad = tamanio / 2;
        std::vector<Valoracion> primeraMitad;
        std::vector<Valoracion> segundaMitad;
        
        for (int i = 0; i < mitad; i++) { primeraMitad.push_back(vector[i]); }
        for (int i = mitad; i < tamanio; i++) { segundaMitad.push_back(vector[i]); }
        
        primeraMitad = ordenacionPorFusion(primeraMitad);
        segundaMitad = ordenacionPorFusion(segundaMitad);
        vector.clear();
        
        for (int i = 0; i < primeraMitad.size(); i++)
            vector.push_back(primeraMitad[i]);
        
        for (int i = 0; i < segundaMitad.size(); i++)
            vector.push_back(segundaMitad[i]);
    }
    
    return vector;
}
\end{lstlisting}



\end{enumerate}

Todo el material incluido en esta memoria y en los ficheros asociados es de mi autoría o ha sido facilitado por los profesores de la asignatura. Haciendo entrega de esta práctica confirmo que he leído la normativa de la asignatura, incluido el punto que respecta al uso de material no original.

\end{document}
