- Función para saber si un nodo esta en una lista.

\lstset{language=C++, texcl=true}
\begin{lstlisting}[frame=single]
bool nodoEnLista(AStarNode* nodo, std::vector<AStarNode*> lista)
{
    return (lista.end() != std::find(lista.begin(), lista.end(), nodo));
}
\end{lstlisting}

- Función para calcular asignar los valores a las celdas, dependiendo de su cercanía
a cada defensa.

\lstset{language=C++, texcl=true}
\begin{lstlisting}[frame=single]
void DEF_LIB_EXPORTED calculateAdditionalCost(float** additionalCost, int cellsWidth,
                                              int cellsHeight, float mapWidth,
                                              float mapHeight, List<Object*> obstacles,
                                              List<Defense*> defenses)
{
    float cellWidth = mapWidth / cellsWidth;
    float cellHeight = mapHeight / cellsHeight;
    
    for(int i = 0 ; i < cellsHeight ; ++i)
    {
        for(int j = 0 ; j < cellsWidth ; ++j)
        {
            Vector3 cellPosition = cellCenterToPosition(i, j, cellWidth, cellHeight);
            additionalCost[i][j] = 0; // Por defecto, valor 0
            
            for (auto d : defenses)
            {
                float distanciaCeldaDefensa = _sdistance(cellPosition, d->position);
                float diff = d->range - distanciaCeldaDefensa;
                
                if(diff < 0) // Si el uco esta dentro del rango de accion de la defensa
                {
                    // Se le da un valor alto para que huya de la defensa
                    // Asi, huye siempre de las defensas mas poderosas siempre
                    additionalCost[i][j] = diff * d->attacksPerSecond * d->damage * d->health;
                }
            }
        }
    }
}
\end{lstlisting}

- Función que se encarga de calcular el camino en cada caso.

\lstset{language=C++, texcl=true}
\begin{lstlisting}[frame=single]
void DEF_LIB_EXPORTED calculatePath(AStarNode* originNode, AStarNode* targetNode,
                                    int cellsWidth, int cellsHeight,
                                    float mapWidth, float mapHeight,
                                    float** additionalCost,
                                    std::list<Vector3> &path)
{
    float min = INF_F; // Distancia minima es el maximo valor para los float en esta maquina
    AStarNode* current = originNode;
    std::list<Vector3> auxiliarPath; // vector auxiliar para ir almacenando la mejor solucion de forma parcial
    
    std::vector<AStarNode*> nodosCerrados;
    
    std::vector<AStarNode*> nodosAbiertos;
    nodosAbiertos.push_back(current); // Se incluye el nodo actual a la lista de abiertos
    
    std::make_heap (nodosAbiertos.begin(), nodosAbiertos.end()); // Se crea montículo con la lista de nodos abiertos
    
    while(!nodosAbiertos.empty())
    {
        // Itera mientras que el nodo actual sea distinto del nodo final
        while(current != NULL && current != targetNode && !nodosAbiertos.empty())
        {
            current = nodosAbiertos.front(); // Se obtiene el ultimo nodo adyacente agregado
            std::pop_heap(nodosAbiertos.begin(), nodosAbiertos.end());
            nodosAbiertos.pop_back(); // Se elimina dicho nodo de la lista de abiertos
            
            nodosCerrados.push_back(current); // Se añade el nodo actual a la lista de cerrados
            for (auto nodo : current->adjacents) // Se añade a la lista de abiertos todos los adyacentes
            {
                if(!nodoEnLista(nodo, nodosCerrados))
                {
                    float distTarget = _sdistance(nodo->position, targetNode->position); // Distancia euclidea, tratada como la manhattan
                    float distParent = _sdistance(nodo->position, current->position); // Distancia con el padre
                    
                    if(!nodoEnLista(nodo, nodosAbiertos))
                    {
                        int i = (int) (nodo->position.y / cellsHeight);
                        int j = (int) (nodo->position.x / cellsWidth);
                        float heu = additionalCost[i][j];  // Se obtiene el valor adicional a la heuristica
                        
                        nodo->H = distTarget + heu; // Heuristica mas su valor adicional
                        nodo->G = distParent + current->G; // Distancia recorrida
                        nodo->F = nodo->G + nodo->H; // Suma de distancia recorrida y heuristica
                        nodo->parent = current; // En este nodo se setea su padre al nodo actual
                        
                        nodosAbiertos.push_back(nodo);
                    } else
                    {
                        if(distParent + current->G < nodo->G)
                        {
                            nodo->G = distParent + current->G;
                            nodo->parent = current;
                        }
                    }
                }
            }
            
            struct CompararAStarNode // Objeto funcion para comparar que AStarNode es mejor
            {
                int operator() (const AStarNode* a, const AStarNode* b) const
                { return a->F < b->F; }
            } f;
            
            std::sort_heap (nodosAbiertos.begin(), nodosAbiertos.end(), f);
        }
        
        if(current == targetNode) // Si el nodo actual es igual al target
        {
            // Se calcula el coste de esta solucion (puede que no sea la mejor)
            float mejorSolucion = 0;
            while(current != originNode && current != NULL)
            {
                if(current->parent != NULL)
                    mejorSolucion += _sdistance(current->position, current->parent->position);
                auxiliarPath.push_front(current->position);
                current = current->parent;
            }
            
            if(mejorSolucion < min) // Si es la mejor solucion encontrada, se impone como la solucion de menor coste
            {
                min = mejorSolucion;
                path = auxiliarPath;
            }
        }
    }
}
\end{lstlisting}
