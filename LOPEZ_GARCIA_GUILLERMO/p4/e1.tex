Respecto al funcionamiento del algoritmo, se ha empleado el algoritmo A*.
Es decir, se tienen dos listas de nodos, cerrados y abiertos, donde
después de analizar un nodo, se incluye dicho nodo a la lista de cerrados
y sus adyacentes a la lista de abiertos.

Se intera a lo largo de la lista de abiertos hasta encontrar un camino o bien,
si se acaba la lista de abiertos y no se encuentra una solución.

La ordenación de los nodos mas prometedores se hace mediante la estructura de un
montículo (proporcionada por la stl).

Una vez encontrada una solución, se sigue iterando en la lista de abiertos hasta
que se vacie o hasta que se encuentre una solución mejor a la obtenida en
primera instancia.

Para ordenar el montículo, se ha usado un objeto función propio del C++11 y
para comprobar si un nodo esta en una lista determinada, se ha usado la siguiente
función:

\lstset{language=C++, texcl=true}
\begin{lstlisting}[frame=single]
bool nodoEnLista(AStarNode* nodo, std::vector<AStarNode*> lista)
{
    return (lista.end() != std::find(lista.begin(), lista.end(), nodo));
}
\end{lstlisting}

También, hay que aclarar que la lista de nodos, se ha modelado con un vector
dinámico de la stl del C++11.

Para terminar, diremos que a los atributos F, G y H de los nodos AStarNode,
se le da un valor dependiendo de la distancia recorrida hasta ese momento
y el valor proveido de la función:

\lstset{language=C++, texcl=true}
\begin{lstlisting}[frame=single]
void DEF_LIB_EXPORTED calculateAdditionalCost(float** additionalCost, int cellsWidth,
                                              int cellsHeight, float mapWidth,
                                              float mapHeight, List<Object*> obstacles,
                                              List<Defense*> defenses)
{
    float cellWidth = mapWidth / cellsWidth;
    float cellHeight = mapHeight / cellsHeight;
    
    for(int i = 0 ; i < cellsHeight ; ++i)
    {
        for(int j = 0 ; j < cellsWidth ; ++j)
        {
            Vector3 cellPosition = cellCenterToPosition(i, j, cellWidth, cellHeight);
            additionalCost[i][j] = 0; // Por defecto, valor 0
            
            for (auto d : defenses)
            {
                float distanciaCeldaDefensa = _sdistance(cellPosition, d->position);
                float diff = d->range - distanciaCeldaDefensa;
                
                if(diff < 0) // Si el uco esta dentro del rango de accion de la defensa
                {
                    // Se le da un valor alto para que huya de la defensa
                    // Asi, huye siempre de las defensas mas poderosas siempre
                    additionalCost[i][j] = diff * d->attacksPerSecond * d->damage * d->health;
                }
            }
        }
    }
}
\end{lstlisting}

Dicha función evalua con un mayor coste las celdas cercanas a las defensas mas poderosas,
así pues, el camino hacia el centro de extracción será mas caro por donde estén las
defensas mas poderosas y mas barato por donde estén las defensas mas débiles.
